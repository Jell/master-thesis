%%%%%%%%%%%%%%%%%%%%%%%%%%%%%%%%%%%%%%%%%%%%%%%%%%%%%%%%%%%%%%%%%%%%%%%%%%%%%%%%%%%%%%%%%%%%%%%%%%%%%%%%%%%%%%%%%%%%%%%%%%%%%%%%%
\noindent
\thesistitle\\
\thesissubtitle\\
Master's Thesis in \textsf{\thesisprogram}\\
\thesisauthor\\
Department of \thesisdepartment\\
Chalmers University of Technology\\

\vspace{1.5cm}
\section*{Abstract}
\addcontentsline{toc}{chapter}{Abstract}

This thesis work was aimed at implementing an entire Augmented Reality system for the iPhone platform. The application developed was designed to display nearby bus stops in Augmented Reality. This paper shows how to estimate the position of an object within six degrees of freedom using the instruments available on modern smart-phones, how to structure a view hierarchy for an Augmented Reality application and how to render objects in 3D using projection matrices.\\

During this study, a Yaws server has also been implemented for the application to retrieve data. The implemented server aggregate and parse data from various online providers such as Västtrafik, the Göteborg's public transport company. On this server runs Erlang code that manages queries to Mnesia and SQLite databases, parallel threads, http requests and caching. A Rails server is also implemented enabling Ruby to be used for database maintenance.\\

\vspace{\baselineskip}
\textbf{Keywords:} iPhone, Augmented Reality, Realtime, AppStore, Ruby, Yaws, Erlang, Objective-C, Mnesia, SQLite


