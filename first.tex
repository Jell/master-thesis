%%%%%%%%%%%%%%%%%%%%%%%%%%%%%%%%%%%%%%%%%%%%%%%%%%%%%%%%%%%%%%%%%%%%%%%%%%%%%%%%%%%%%%%%%%%%%%%%%%%%%%%%%%%%%%%%%%%%%%%%%%%%%%%%%
\chapter{Introduction}
\label{cha:introduction}

Augmented Reality is a dream coming true. The concept itself is not brand new and has been developed and studied for many years, but modern technologies and devices are now allowing Augmented Reality to be implemented almost anywhere and be used on a daily basis in a vast range of applications from advertising to medical assistance.\\

In the mean time, the smart-phones of the last generation have been equipped with sensors that locate the user and enable cleaver and innovative interactive experiences. The iPhone is one of those phones, and taking advantage of its hardware to build an Augmented Reality experience is an interesting challenge.\\

This master thesis focuses on the design of an application of Augmented Reality on iPhone 3G-S and its implementation from both a Client and Server point of view. This paper presents the work that has been done, starting with the methodology that was used (Chapter \ref{cha:methodology}).\\

At first an overview over the basic principles of Augmented Reality (Chapter \ref{cha:basic_principles}) will help understanding what it is, how it works and how it can be used. Then we will have a look at the "client side" of the application (Chapter \ref{cha:on_the_client_side}) to see the definition of the application from a user's point of view, and we will go through the implementation on the iPhone to understand the underlying code structure. We will also have a look at the server implementation (Chapter \ref{cha:on_the_server_side}) to get an idea of the architecture that is required to provide data in real-time.\\

Finally, we will describe the results achieved with the current implementation (Chapter \ref{cha:conclusion}), and a view on further improvements will finalize the thesis.\\

This thesis has been carried at ICE House AB, a development company for web enterprises based in Göteborg. 
