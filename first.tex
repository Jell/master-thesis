%%%%%%%%%%%%%%%%%%%%%%%%%%%%%%%%%%%%%%%%%%%%%%%%%%%%%%%%%%%%%%%%%%%%%%%%%%%%%%%%%%%%%%%%%%%%%%%%%%%%%%%%%%%%%%%%%%%%%%%%%%%%%%%%%
\chapter{Introduction}
\label{cha:introduction}

Augmented Reality is a dream coming true. Of course, the concept itself is not brand new and has been used and studied for many years, but modern technologies and devices now enable Augmented Reality to be implemented relatively painlessly and be used on a day-to-day basis, in a vast range of applications from advertising to medical assistance.\\

At the same time, the last generation smartphones are equipped with sensors that locates its users and can interact with them in cleaver and innovative ways. The iPhone is one of those. That is why Augmented Reality on smartphones, and especially on iPhone, has been growing fast for the last couple of years.\\

This master thesis focuses on the design of an application of Augmented Reality on iPhone 3G-S and its implementation from both a Client and Server point of view.\\

At first an overview over the basic principles of Augmented Reality (chapter \ref{cha:basic_principles}) will help understanding Augmented Reality, how it works and how it can be used. Then we will have a look at the Client side of the application (chapter \ref{cha:on_the_client_side}) to see what the application is from a user's point of view, and we will go through the implementation on the iPhone to understand the underlying code structure. We will also have a look at the Server implementation (chapter \ref{cha:on_the_server_side}) to get an idea of the require architecture to provide data in real-time. Finally, we will describe the results achieved with the current implementation (chapter \ref{cha:conclusion}), and a view on further improvements will finalize the thesis.